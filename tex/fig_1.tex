\documentclass{standalone}

\usepackage{tikz}
\usetikzlibrary{shapes,arrows}
\usepackage{amsmath,bm,times}
\newcommand{\mx}[1]{\mathbf{\bm{#1}}} % Matrix command
\newcommand{\vc}[1]{\mathbf{\bm{#1}}} % Vector command

\begin{document}
%\pagestyle{empty}
%	\centering

% Define layers to draw the block diagram
\pgfdeclarelayer{background}
\pgfdeclarelayer{foreground}
\pgfsetlayers{background,main,foreground}

% Define colors
\colorlet{a_gray}{black!20}
\colorlet{a_brown}{black!28!brown!16}
\colorlet{a_pale_brown}{black!16!brown!10}
\colorlet{a_pale_gray}{black!14}

% Define a few styles and constants
\tikzstyle{block} = [draw, text width=8em, text centered, minimum height=2.2em]

\tikzstyle{inputs}=[block, fill=a_pale_brown]
\tikzstyle{outputs} = [block, fill=a_pale_brown]
\tikzstyle{function} = [block, text width=6em, fill=a_brown]
\tikzstyle{models} = [block, text width=6em, fill=a_gray, rounded corners]
\tikzstyle{sub_models} = [block, text width=4em, fill=a_pale_gray]
 \tikzstyle{sub_data} = [block, text width=4em, fill=a_pale_brown]
    
%anotate arrows    
\tikzstyle{ann} = [above, text width=5em]

% Define some distances
\def\blockdist{4}
\def\external{8}
\def\edgedist{2.5}

\tikzset{>=latex}


% Make some macros
\newcommand{\InputBlock}[7]{% node1, hight, label1, node2, label2, arrow1, arrow2
	\path (model)+(-\external,#2) node (#1) [inputs] {#3};
	\path (model)+(-\blockdist,#2) node (#4) [function] {#5};
	\path [draw, ->] (#1.east |- #1.east) -- node [above, pos=0.2,] {\tiny{#6}} (#4.west |- #4.west);
    \path [draw, ->] (#4.east |- #4.east) -- node [above] {\tiny{#7}} (model.west |- #4.west);
}

\newcommand{\OutputBlock}[7]{% node1, hight, label1, node2, label2, arrow1, arrow2
	\path (model)+(\blockdist,#2) node (#1) [function] {#3};
	\path (model)+(\external,#2) node (#4) [outputs] {#5};
	\path [draw, ->] (model.east |- #1.west) -- node [above] {\tiny{#6}} (#1.west |- #1.west);
	\path [draw, ->] (#1.east |- #4.west) -- node [above] {\tiny{#7}} (#4.west |- #4.west);
}








\begin{tikzpicture}
	
	%xarray dataset, in center of fig
    \node (model) [models, minimum height=5cm, minimum width=4cm, text depth=3.5cm] {xarray Dataset};
    
    
    		\path (model.140)+(1.3,0) node () [] {};
   		 	\path[draw=black] (-1,1.3) -- (-1,0.8) -- (-0.5,0.8);
    		\path[draw=black] (-0.8,1) -- (-1,0.8);
    
    		\path (model.150)+(1.02,-0.6) node (sub_model) [sub_models] {};
    		\path (model.150)+(1.08,-0.7) node (sub_model) [sub_models] {};
			\path (model.150)+(1.14,-0.8) node (sub_model) [sub_models] {};
    		\path (model.150)+(1.2,-0.9) node (sub_model) [sub_models] {DataArray};
    
    		\path (model.150)+(1.2,-2) node (sub_data) [sub_data] {meta data};
    
   % inputs
   
   
   \InputBlock{raster_in}{1}{Raster data}{raster_proc}{Affine transformation}{geoTiff}{numpy}
   \InputBlock{vector_in}{0}{Vector data}{vector_proc}{Rasterization}{geoTiff}{numpy}

   \InputBlock{grid_in}{-1}{Grid data}{grid_proc}{Interpolation}{netCDF}{numpy}
   \InputBlock{ascii_in}{-2}{ASCII data}{ascii_proc}{Interpolation}{ascii}{numpy}
   
   \OutputBlock{raster_out_proc}{1}{rasterio}{raster_out}{Raster}{numpy}{geoTiff}
   \OutputBlock{fig_out_proc}{0}{matplotlib}{fig_out}{Map}{numpy}{pdf}
   \OutputBlock{3D_out_proc}{-1}{mayavi}{3D_out}{3D viz}{numpy}{screen / pdf}
   \OutputBlock{grid_out_proc}{-2}{xarray}{grid_out}{Grid export}{xarray}{netCDF}
   
   
 %  \OutputBlock{grid_out}{0}{Map}{netCDF}{}{}
   %\OutputBlock{grid_out}{-1}{3D viz}{}{}{}
  % \OutputBlock{grid_out}{-2}{Raster}{geoTiff}{}{}
       
 
 
   
%   
%   \path (model.150)+(-\external,1) node (raster_in) [input] {Raster data};
%   \path (model.150)+(-\blockdist,1) node (raster) [function] {Affine transformation};
%   \path [draw, ->] (raster_in) -- node [above, pos=0.2,] {\tiny geoTiff} (raster.west |- raster.west);
%   \path [draw, ->] (raster) -- node [above] {\tiny{numpy}} (model.west |- raster.west);
%    
%    
%   \path (model.150)+(-\external,0) node (raster_in) [input] {Raster data};
%   \path (model.150)+(-\blockdist,0) node (raster) [function] {Affine transformation};
%   \path [draw, ->] (raster_in) -- node [above, pos=0.2,] {\tiny geoTiff} (raster.west |- raster.west);
%   \path [draw, ->] (raster) -- node [above] {\tiny{numpy}} (model.west |- raster.west);
%   
%   
%
%   
%	\path (model.150)+(-\external,-2) node (raster_in) [input] {Raster data};
%	\path (model.150)+(-\blockdist,-2) node (raster) [function] {Affine transformation};
%	\path [draw, ->] (raster_in) -- node [above, pos=0.2,] {\tiny geoTiff} (raster.west |- raster.west);
%	\path [draw, ->] (raster) -- node [above] {\tiny{numpy}} (model.west |- raster.west);
%
%
%    
%    
    
    

    
    \path (model)+(0,-3.2) node (attributes) [models, minimum height=2em] {Instance attributes};    
    \path [draw, <->] (model) -- node {} (model.south |- attributes.north);

    \path (model)+(0,3.4) node (python) [models, minimum height=2em] {Python};    
    \path [draw, <->] (model) -- node {} (model.north |- python.south);
   
   \path (model)+(0,4.6) node (python_api) [models, minimum height=2em] {Python api};    
	\path [draw, <->] (python_api) -- node {} (python_api.south |- python.north);
 
    \path (attributes)+(0,-1) node (grid_attr) [models, minimum height=2em] {Grid attributes};    
    
    \path (grid_attr.south west)+(-0.6,-0.8) node (class) {Python class Grid};
    
    
    
%%    \path (raster.south)+(0,-4.4) node (staal) [function] {Convolution of vector lines};
   %% \path (raster.south)+(-4.7,-4.4) node (staal_in) [input] {Vector data};
   %% \path [draw, ->] (staal_in) -- node [above, pos=0.3,] {\tiny e.g. shapefile} (staal.west |- staal.west);
    
%%    \path (model.150)+(-\blockdist,-5) node (staal) [function] {Interactive visualization};
    
    
    
    
   %  \draw [->] (model) -- node [attributes] {} + (\edgedist,0) 
    %node[right] {};

    
   % \draw [->] (model.20) -- node [ann] {pdf} + (\edgedist,0) 
    %    node[right] {Figure};
    %\draw [->] (model.-25) -- node [ann] {GeoTiff} + (\edgedist,0)
        %node [right] {GIS};
    %\draw [->] (model.-50) -- node [ann] {global png} + (\edgedist,0) 
     %   node[right] {Morse (ref)};
    
    \begin{pgfonlayer}{background}
        % Compute a few helper coordinates
%%%        \path (raster.west |- raster.north)+(-0.3,0.3) node (a) {};
      %%%  \path (class.south -| model.east)+(+0.3,-0.3) node (b) {};        
        
        \path[fill=brown!30,rounded corners, draw=black!50]
     (raster_proc.west |- model.north)+(-0.3,0.3) rectangle (class.south -| grid_out_proc.east)+(+0.3,+0.3) ;
            
        
   
            
          
        %\path (raster.north west)+(-.2,0.2) node (a) {};
        %\path (Import.south -| raster.east)+(+0.2,-0.2) node (b) {};
        
        
       % \path[fill=green!10,rounded corners, draw=black!50, dashed]
         %   (raster.west |- raster.north)+(-0.2,0.2) rectangle (raster.east -| raster.south)+(+0.2,+0.2);
            
       %\path[fill=blue!10,rounded corners, draw=black!50, dashed]
      %(staal.west  |- staal.north)+(-0.5,0.5) rectangle (staal.east  |- staal.south)+(-0.5,-0.5);
      
            
    \end{pgfonlayer}
\end{tikzpicture}



\end{document}
