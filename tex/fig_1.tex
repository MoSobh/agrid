\documentclass[tikz]{standalone}

\usepackage{tikz}
\usetikzlibrary{calc}
\usetikzlibrary{shapes,arrows}
\usepackage{amsmath,bm,times}
\newcommand{\mx}[1]{\mathbf{\bm{#1}}} % Matrix command
\newcommand{\vc}[1]{\mathbf{\bm{#1}}} % Vector command

\begin{document}
%\pagestyle{empty}
%	\centering

% Define layers to draw the block diagram
\pgfdeclarelayer{background}
\pgfdeclarelayer{foreground}
\pgfsetlayers{background,main,foreground}

% Define colors
\colorlet{a_gray}{black!18}
\colorlet{a_brown}{black!28!brown!16}
\colorlet{a_pale_brown}{black!16!brown!10}
\colorlet{a_pale_gray}{black!14}

% Define a few styles and constants
\tikzstyle{block} = [draw, text width=8em, text centered, minimum height=2.2em]

\tikzstyle{inputs}=[block, fill=a_pale_brown]
\tikzstyle{outputs} = [block, fill=a_pale_brown]
\tikzstyle{function} = [block, text width=6em, fill=a_brown]
\tikzstyle{models} = [block, text width=6em, fill=a_gray, rounded corners]
\tikzstyle{sub_models} = [block, text width=4em, fill=a_pale_gray, minimum width=4em]
\tikzstyle{sub_coords} = [block, text width=3.2em, minimum width=4em, minimum height=4em, fill=a_pale_gray]
\tikzstyle{sub_data} = [block, text width=4em, fill=a_pale_gray]
    
%anotate arrows    
\tikzstyle{ann} = [above, text width=5em]

% Define some distances
\def\blockdist{4}
\def\external{8}
\def\edgedist{2.5}

\tikzset{>=latex}

% Make some macros
\newcommand{\InputBlock}[7]{% node1, hight, label1, node2, label2, arrow1, arrow2
	\path (model)+(-\external,#2) node (#1) [inputs] {#3};
	\path (model)+(-\blockdist,#2) node (#4) [function] {#5};
	\path [draw, ->] (#1.east |- #1.east) -- node [above, pos=0.4,] {\tiny{#6}} (#4.west |- #4.west);
    \path [draw, ->] (#4.east |- #4.east) -- node [above] {\tiny{#7}} (model.west |- #4.west);
}

\newcommand{\OutputBlock}[7]{% node1, hight, label1, node2, label2, arrow1, arrow2
	\path (model)+(\blockdist,#2) node (#1) [function] {#3};
	\path (model)+(\external,#2) node (#4) [outputs] {#5};
	\path [draw, ->] (model.east |- #1.west) -- node [above] {\tiny{#6}} (#1.west |- #1.west);
	\path [draw, ->] (#1.east |- #4.west) -- node [above] {\tiny{#7}} (#4.west |- #4.west);
}


\begin{tikzpicture}
\sffamily
	%xarray dataset, in center of fig
    \node (model) [models, minimum height=5cm, minimum width=4cm, text depth=0cm] {xarray Dataset};
    
    		\path (model.140)+(1.3,0) node () [] {};
   		 	\path[draw=black] (0.5,-1) -- (0.5,-2) -- (1.5,-2);
    		\path[draw=black] (0.5,-2) -- (1,-1.5);
    		\path (1.1,-1.2) node () {dims};
    
    		\path (model.150)+(1.02,0.7) node (sub_model) [sub_models] {};
    		\path (model.150)+(1.08,0.6) node (sub_model) [sub_models] {};
			\path (model.150)+(1.14,0.5) node (sub_model) [sub_models] {};
    		\path (model.150)+(1.2,0.4) node (sub_model) [sub_models] {DataArray};
    		
    		\path (model.150)+(1.2,-2.8) node (sub_data) [sub_data] {meta data};
    
    		\path (model.150)+(3,0.5) node (coords) [sub_coords] {\small{coords}};
    

   % inputs
   \InputBlock{raster_in}{1}{Raster data}{raster_proc}{Affine transformation}{geoTiff}{numpy}
   \InputBlock{vector_in}{0}{Vector data}{vector_proc}{Rasterization}{geoTiff}{numpy}
   \InputBlock{grid_in}{-1}{Grid data}{grid_proc}{Interpolation}{netCDF}{numpy}
   \InputBlock{ascii_in}{-2}{ASCII data}{ascii_proc}{Interpolation}{ascii}{numpy}
   
   % outputs
   \OutputBlock{raster_out_proc}{1}{rasterio}{raster_out}{Raster}{numpy}{geoTiff}
   \OutputBlock{fig_out_proc}{0}{matplotlib}{fig_out}{Map}{numpy}{pdf}
   \OutputBlock{3D_out_proc}{-1}{mayavi}{3D_out}{3D viz}{numpy}{screen / pdf}
   \OutputBlock{grid_out_proc}{-2}{xarray}{grid_out}{Grid export}{xarray}{netCDF}
   
	%xarray    
    \path (model)+(0,-3.5) node (attributes) [models, minimum height=2em] {instance \\ attributes};    
    \path [draw, <->] (model) -- node {} (model.south |- attributes.north);

    \path (model)+(0,3.4) node (python) [models, minimum height=2em] {Python};    
    \path [draw, <->] (model) -- node {} (model.north |- python.south);
   
   \path (model)+(0,4.6) node (python_api) [models, minimum height=2em] {Python api};    
	\path [draw, <->] (python_api) -- node {} (python_api.south |- python.north);

    \path (attributes)+(0,-1) node (grid_attr) [models, minimum height=2em] {class \\ attributes};    
    \path (grid_attr.south)+(-0,-0.6) node (class) {Python class Grid()};
    
    \path ($(raster_in)+(-2.3,3)$) node (download) [function] {Download};
    \path ($(download.north)+(0,+0.6)$) node (acc) {Accessory methods};
    
    \path [draw, ->] (download.south) --++(0:0mm)|- (ascii_in.west);
    \path [draw, ->] (download.south) --++(0:0mm)|- (vector_in.west);
    \path [draw, ->] (download.south) --++(0:0mm)|- (grid_in.west);
    \path [draw, ->] (download.south) --++(0:0mm)|- (raster_in.west);
    
    \path ($(raster_out)+(0,4)$) node (staal) [function] {Staal et al };
    \path ($(raster_out)+(0,3)$) node (morse) [function] {Morse et al};
   	\path ($(staal.north)+(0,+0.6)$) node (studies) {Feature methods};
    
    \path ($(morse.south)+(0,-1)$) node (png) [outputs] {Global png};

    
    \path [draw, ->] (download.south) --++(0:0mm)|- (grid_in.west);
    \path [draw, <->] ($(model.east)+(0,2.1)$) --++(0:0.8)|- (staal.west);
    \path [draw, ->] ($(model.east)+(0,1.8)$) --++(0:1.2)|- (morse.west);
    \path [draw, ->] (morse.south) -- node {} (png.north);
    
    
    \begin{pgfonlayer}{background}  
		\path[fill=brown!30,rounded corners, draw=black!50]
     		($(raster_proc.west |- model.north)+(-0.2,0.2)$) rectangle ($(raster_out_proc.east |- class.south)+(0.2,0)$);
     		
     	\path[fill=brown!30,rounded corners, draw=black!50]
     	($(acc.west |- acc.north)+(-0.2,0.2)$) rectangle ($(acc.east |- download.south)+(0.2,-0.2)$);
     		
     	\path[fill=brown!30,rounded corners, draw=black!50]
     	($(studies.west |- studies.north)+(-0.2,0.2)$) rectangle ($(studies.east |- morse.south)+(0.2,-0.2)$);
     	
     		
     		
    \end{pgfonlayer}
\end{tikzpicture}



\end{document}
